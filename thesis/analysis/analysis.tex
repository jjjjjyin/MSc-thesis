\chapter{Chain conformation analysis}\label{chap_analysis}
\graphicspath{{./analysis/graphs/}}

\section{Polymer crystallization}

\subsection{Polymer crystal models and theories}

Material systems naturally tend to stay in a lower energy state. Therefore a regular liquid composed of small molecules, upon cooling, will transition into the solid state (the ground state), becoming either amorphous or crystalline. The true ground state for a polymer corresponds to the situation of all monomers extended and aligned parallel with each other. However, since the chains are relatively long and randomly aligned in the liquid state, it is much more difficult for the polymer to achieve this ground state. Instead, polymers can still crystallize under proper conditions, adopting more complex structures rather than the ideal crystalline state. This process does not only depend on thermodynamics, but kinetics as well.

Before examining long polymers chains, oligomers (especially linear ones) are a simpler yet close enough example in terms of crystallization. Based on X-ray crystallography results, oligomer crystals adopt a structure of stacked layers, with each layer composed of chains standing up perpendicular to the layer surface (Figure \ref{fig:oligomercrystallization}) \cite{Strobl2007a}. Among neighbouring layers, end-groups of the chains form an amorphous phase at the interfaces (not shown in the drawing).

\begin{figure}[H]
\center
\includegraphics[width=0.4\linewidth]{oligomercrystallization}
\caption[Oligomer crystal structure.]{Oligomer crystal structure. Figure source: "The physics of polymers: Concepts for understanding their structures and behaviour" by Gert Strobl, 2007 \cite{Strobl2007a}.}
\label{fig:oligomercrystallization}
\end{figure}

For crystallization of polymers with higher molecular weights, however, it is impossible for the chains to completely disentangle, which would require an extremely high energy and a very long time. Limited by the nature of polymers themselves, the chains align into local crystalline domains, with some unresolved entanglements left as amorphous phases in between. Similar to oligomers, end-groups are also part of the amorphous phase. Therefore, polymer crystals are called semicrystalline crystals.

\subsubsection{Fringed micelle model}

In order to describe semicrystalline polymer crystal structures in further details, different models and theories have been proposed. One of the earliest (in 1930) models is the fringed micelle model \cite{HerrmannKGerngrossO1930}, as shown in Figure \ref{fig:fringedmicelle}. In this model, both the crystalline phase and the amorphous phase are present, with the crystallites existing as local domains. The micelles of crystalline parts have sizes much smaller than the chain lengths, so a single polymer chain is believed to be able to pass through several micelles, thus binding them together.

%Herrmann, K.; Gerngross, O.; Abitz, W. Zur rontgenographischen Strukturforschung des gelatinemicells. Z. Phys. Chem. B 1930, 10, 371–394.

\begin{figure}[H]
\center
\includegraphics[width=0.3\linewidth]{fringedmicelle}
\caption[Fringed micelle model of polymer crystal structure.]{Fringed micelle model of polymer crystal structure. Figure source: "Principles of polymer chemistry" by Paul J. Flory, \textit{Cornell University Press}, 1953 \cite{Flory1953}.}
\label{fig:fringedmicelle}
\end{figure}

However, there were some problems with the fringed micelle model. According to the calculation of free energy, it was found that there would be a large conformational entropy loss for the amorphous chains if this model was true \cite{Flory1962}. In addition, experimentalists observed evidences of large crystallites - "spherulites", which have a strong preference in terms of the alignment of chains, and are highly symmetrical instead of a random distribution of crystallites \cite{Geil1964}. Together with some other flaws and contradictions found with the model itself \cite{Zachmann1967,Zachmann1969}, people began to doubt the fringed micelle model and tried to find other ways to describe semicrystalline polymer crystals.

\subsubsection{Folded chain model} \label{sec: folded chain model}

As the fringed micelle model was being questioned, some crucial experimental observations led to the birth of a new model of polymer crystals---the folded chain model. The concept of chain-folding was first proposed by Storks \cite{Storks1938}. He observed unstretched films of gutta-percha through electron diffraction measurements, and found that the films were composed of large crystallites with the chain axis perpendicular to the film surfaces. The thickness of the films were much smaller than the chain lengths, which led to Storks' proposal that the chains need to fold themselves inside the film.  At that time (1938), the fringed micelle model was dominating the directions of polymer crystallization research, so his results and proposal did not receive much attention. Later, several researchers \cite{JACCODINE1955,Till1957,Keller1957} studied polymer single crystals and found that they have smooth surfaces, with heights of about 10 nm, which was also much smaller than the chain lengths used in those studies. The chains are believed to fold themselves back and forth in the layer, and when the polymer solution concentration is high enough, or when the polymer crystallizes from a melt, multiple layers stack together to form a crystal, with each layer called a lamella. These observations helped the development of the folded chain model, which from then on became the most widely accepted model of polymer crystals. 

Now it is clear that the chains fold in crystals. The next step is to determine the way the chains fold. After the chain gets to the amorphous interface and folds back on itself, it is not clear where it re-enters the lamella. There have been a large number of studies on this \cite{Kovacs1975,Yoon1979,Keller1979} and two major models have been proposed: adjacent re-entry and random re-entry.

As the adjacent re-entry model describes, after a chain escapes the lamella and makes a fold, it turns right back and inserts into the neighboring site. In this way, the lamellae created have relatively smooth surfaces, as Figure \ref{fig:adjacent} shows.

\begin{figure}[H]
\center
\includegraphics[width=0.4\linewidth]{adjacent}
\caption[Adjacent re-entry model of polymer crystals.]{Adjacent re-entry model of polymer crystals. Figure source: "On the Morphology of the Crystalline State in Polymers" by P. J. Flory, \textit{J. Am. Chem. Soc.}, 1962 \cite{Flory1962}.}
\label{fig:adjacent}
\end{figure}

The random re-entry model is also known as the switchboard model. Instead of folding right back into the neighbouring site on the same lamella as in the adjacent re-entry, a chain that emanated from the lamellar surface could either float on the interface and walk into a further site, or even bridge across to another lamella crystal (not shown in Figure \ref{fig:random}), which leads to a completely random arrangement on the interfaces of lamellae and amorphous regions (Figure \ref{fig:random}).

\begin{figure}[H]
\center
\includegraphics[width=0.4\linewidth]{random}
\caption[Random re-entry model of polymer crystals.]{Random re-entry model of polymer crystals. Figure source: "On the Morphology of the Crystalline State in Polymers" by P. J. Flory, \textit{J. Am. Chem. Soc.}, 1962 \cite{Flory1962}.}
\label{fig:random}
\end{figure}

The adjacent re-entry model would result in a much more thermodynamically favourable conformation with a lower entropy. However, in a real situation of polymer crystallization, twisting, misalignment, and entanglements of the long chains prevent them from relaxing and aligning perfectly in regular folds within the time available. Instead, regions consisting of too many entanglements are more likely to be shifted to the surfaces and contribute to the amorphous phase \cite{Strobl2007a}. Taking both thermodynamics and kinetics factors into account, real polymer crystal lamellae are normally composed of both adjacent re-entries and random re-entries. It should be noted that in real cases, chain-folding also depends on more factors (e.g. chain lengths, flexibility of chains, crystallization temperature, cooling rate, chain defects).
 
\subsection{Thermodynamics of polymer crystallization} \label{Thermodynamics of polymer crystallization}

In terms of thermodynamics, the most favourable state for polymers is the state with the lowest possible Gibbs free energy $G$. $G$ is lower for a melt than for crystals at high temperatures, while it is lower for crystals than for melt at low temperatures. The equilibrium melting point $T_{m}^{\infty}$ is defined as the temperature at which the liquid state and the solid state have the same free energy. Therefore, the change in the Gibbs free energy, $\Delta G$, is equal to zero during melting or crystallization at thermodynamic equilibrium:

\begin{equation}
\label{eqn_deltaG}
\Delta G = \Delta H - T_{m}^{\infty} \Delta S = 0
\end{equation}

\begin{equation}
\label{eqn_Tminfinity}
T_{m}^{\infty} = \dfrac{\Delta H}{\Delta S}
\end{equation}

In practical cases, the crystallization temperature $T_{c}$ is always lower than the melting temperature $T_{m}$, and their difference is defined as the supercooling $\Delta T$. This is mainly due to the nucleation and growth mechanism during crystallization. A nucleus must be present to initialize the growth of a crystal, and when there is no present nuclei, the temperature is able to keep decreasing until the melt itself starts a primary nucleation. This mechanism will also be further discussed in Chapter \ref{chap_growth}. $\Delta T$ of polymers can be as large as 20 to 30 K, resulting from the metastable chain-folding nature of polymers \cite{Hu2013}.

Equation \ref{eqn_Tminfinity} tells us that the equilibrium melting temperature depends on both the enthalpy and the entropy of the system. However, the effect of surface energy and crystal size has not been considered. For a real polymer crystal, the shape and size of the lamella will directly affect its melting point, and this effect can be examined through thermodynamics.

Let us start with an infinitely large, perfect crystal, that from a conventional thermodynamic viewpoint is considered not to involve surface energy. Therefore its melting point is to be $T_{m}^{\infty}$.

\begin{figure}[H]
	\center
	\includegraphics[width=0.4\linewidth]{lamella}
	\caption{Schematic drawing of a polymer crystal lamella.}
	\label{fig:lamella}
\end{figure}

Now assume a lamella (Figure \ref{fig:lamella}) with length $a$, width $b$, and height $l$, where $a \gg l$, and $b \gg l$. The surface energy per unit area of the top and bottom surfaces is $\sigma_{e}$, and the surface energy per unit area of the side surfaces is $\sigma$. This lamella with the finite size effect could be considered as a quasi-two dimensional object with one-dimensional confinement \cite{Zhang2016a}. The free energy per unit mass on melting is $\Delta g$, and the total free energy $\Delta G$ on melting consists of the energy required to create new surfaces and the energy of fusion for the bulk:

\begin{equation}
\label{eqn_delta G lamella}
\Delta G = 2(a+b)l\sigma + 2ab\sigma_{e} - abl\Delta g
\end{equation}

\noindent
and with $\sigma_{e}\gg\sigma$, $a\gg l$, $b\gg l$, the total free energy is then:

\begin{equation}
\label{eqn_delta G lamella reduced}
\Delta G = 2ab\sigma_{e} - abl\Delta g
\end{equation}

At the melting temperature $T_{m}$, $\Delta G = 0$, which leads to:

\begin{equation}
\label{eqn_delta g lamella}
\Delta g (T_{m}) = \dfrac{2\sigma_{e}}{l}
\end{equation}

Once again, for an infinitely large crystal, we have:

\begin{equation}
\label{eqn_delta g large}
\Delta g (T_{m}^{\infty}) = \Delta h (T_{m}^{\infty}) - T_{m}^{\infty}\Delta s (T_{m}^{\infty}) = 0
\end{equation}

Assuming between $T_{m}$ and $T_{m}^{\infty}$, the enthalpy and entropy could be treated as invariant, we further have:

\begin{equation}
\label{eqn_delta g}
\Delta g (T_{m}) = \Delta h (T_{m}) - T_{m}\Delta s (T_{m})
\end{equation}

Combining Equation \ref{eqn_delta g large} and Equation \ref{eqn_delta g}, we are able to generate:

\begin{equation}
\label{eqn_delta g 2}
\Delta g (T_{m}) = \Delta h (T_{m}) - T_{m}\dfrac{\Delta h (T_{m})}{T_{m}^{\infty}}
\end{equation}

Now with Equation \ref{eqn_delta g lamella} and Equation \ref{eqn_delta g 2}, we finally obtain the relation between the thickness of a lamella and its melting temperature:

\begin{equation}
\label{eqn_GT}
T_{m} = T_{m}^{\infty} (1 - \dfrac{2\sigma_{e}}{l \Delta h})
\end{equation}

\noindent
which is the well-known Gibbs Thomson equation. It has been applied to many polymers with linear structure and has proved to provide reliable predictions of the melting temperature as a function of lamella thickness \cite{KojiYamada2003}. With a larger thickness, the finite size effect is weaker, and the melting temperature $T_{m}$ of the lamella is closer to the equilibrium melting temperature $T_{m}^{\infty}$.

\section{PEO crystallization}

In terms of crystallization, PEO is one of the most intensively studied polymers, together with polyethylene and the n-alkanes. With linear structures, these polymers all crystallize very easily. As a semicrystalline polymer, PEO chains fold into lamellar structures during crystallization, and multiple lamellae stack up to form the whole crystal \cite{Arlif1966}. In our case, we focus on low molecular weight PEO, so crystallization should be even easier since the chains are relatively short and thus need to fold fewer times. When the number of folds changes, the thickness of the lamella varies, which has a direct influence on the melting temperature of the crystal lamella, as discussed in the previous section.

\subsection{Crystal structure} \label{crystalstructure}

PEO crystals have monoclinic unit cells, with the chains adopting a structure of 7/2 helix with trans-gauche-trans conformation. In this conformation, seven monomeric units form two periods of the helix, which is 1.93 nm long \cite{Yoshihara1964}. As shown in Figure \ref{fig:PEOhelix}, every bond is rotated by a certain angle with respect to the c-axis (vertical axis) of the lamella, and the projection length of one monomer on the c-axis, $l_{c}$ (or $h$ as in the figure), is 0.278 nm \cite{Takahashi1973}.

\begin{figure}[H]
\center
\includegraphics[width=0.5\linewidth]{helix}
\caption[7/2 helix structure (a) and its radial projection (b).]{7/2 helix structure (a) and its radial projection (b). Circles with numbers represent monomers. Pitch length, P, and unit length, h, represent the axial lengths of one helical turn and one monomer, respectively. $\theta$ represents the angle between two monomers around the helical axis, and $r$ represents the helix radius. Figure source: "Revisiting the Molecular Structure of Collagen" by Kenji Okuyama, \textit{Connect. Tissue Res.}, 49(5):299-310, 2008 \cite{Okuyama2008}.}
\label{fig:PEOhelix}
\end{figure}

Low molecular weight PEO fractions, or PEO oligomers, crystallize with chains folded a small number of times, or even fully extended \cite{Kovacs1975,Kovacs1977}. The number of folds depends on many factors including crystallization temperature, chain length, and cooling rate. The thickness of the lamella $L$ is thus determined by the number of folds $n$ and the chain length $\lambda$:

\begin{equation}
\label{eqn_thickness}
L = \dfrac{\lambda}{1+n} = \dfrac{N l_{c}}{1+n}
\end{equation}

\noindent
where $N$ is the number of monomers in a chain, or the degree of polymerization. In the special case of fully extended chains, $n = 0$, and the thickness of the lamella is equal to the chain length. The lower molecular weight limit for a PEO chain to adopt the folded configuration is reported to be 2000 $g/mol$, which correspond to 45 monomers \cite{Kovacs1975}. Chains shorter than this have been believed to have the extended configuration only.

In terms of chain folding, we have also discussed the two different chain re-entry models in \ref{sec: folded chain model}: adjacent re-entry and random re-entry. In the case of PEO oligomers, the chains are relatively short, so they are easier to get aligned, and we could expect more adjacent re-entries in the lamellae. On each chain fold on the lamellar surfaces of low molecular weight PEO, it costs 3.5 monomers on average to complete a 180-degree turn, which correspond to half of the 7/2 helix \cite{Buckley1976}.

\subsection{Melting points of PEO oligomers} \label{Tm and Yeates}

The Gibbs Thomson equation (Equation \ref{eqn_GT}), enables one to build the relation between melting points and other physical parameters of a polymer crystal. In order to be consistent with other research on PEO melting transitions, here we make some modifications to the original equation:

\begin{equation}
\label{eqn_GT2}
T_{m} = T_{m}^{\infty} (1 - \dfrac{2SV}{L \Delta H})
\end{equation}

\noindent
where $S$ is the surface free energy of the interface between the crystalline and the amorphous phase, and $V$ is the molar volume of a crystallizable repeat unit \cite{Pfefferkorn2011}. In this equation, $T_{m}^{\infty}$, $V$, and $\Delta H$ are constants that have been determined for PEO.

Melting transitions of PEO have been intensively studied through various experimental methods and from different theoretical aspects. One of the fundamental studies is of particular interest to us and is worth being reviewed. Monodisperse PEO oligomers with a degree of polymerization ranging from 9 to 45 were produced through step-wise syntheses by Yeates \textit{et al} \cite{Yeates1984}. Melting points of these fractions were measured, and compared to those of commercially available samples, which were much more polydisperse. Their results are shown in Figure \ref{fig:Yeates}.

\begin{figure}[H]
\center
\includegraphics[width=0.35\linewidth]{Yeates}
\caption[Melting points vs degree of polymerization for monodisperse (black dots) and polydisperse (empty boxes) PEO oligomer samples.]{Melting points vs degree of polymerization for monodisperse (black dots) and polydisperse (empty boxes) PEO oligomer samples ($n$ was used here in place of $N$). Figure source: "Ethylene glycol oligomers" by Stephen G. Yeates \textit{et al}, \textit{Makromol. Chemie}, 185(8):1559-1563, aug 1984 \cite{Yeates1984}.}
\label{fig:Yeates}
\end{figure}

Melting points of monodisperse samples are notably higher than those of polydisperse samples in general, and the difference is especially large for small $N$ values. This observation indicates that polydispersity has a big influence on melting temperature, and in fact motivated us to conduct crystallization experiments with our purified samples, so that we could further investigate this phenomenon.

As a matter of fact, this observation has attracted much attention from researchers. One of the explanations that has been proposed suggests that the $T_{m}$ difference could be related to the chain end-groups \cite{Percec1989}. In a relatively monodisperse sample, chains have roughly the same length, which makes it easier to create a smooth lamellar surface, and the end-groups would be incorporated in the crystalline array. However, in a polydisperse sample, the distribution of chains results in a more disordered lamellar surface, so some of the end-groups have to be incorporated in the amorphous phase. Because of the difference in the incorporation of chains ends, polydisperse crystals would have lower crystallinity and higher entropy, which leads to a higher melting temperature.

\section{Basics of differential scanning calorimetry}

\subsection{Phase transitions in polymers}

Phase transitions are important to characterize the properties of a given polymer. In regular materials, phase transitions normally refer to the transitions between solid, liquid, and gaseous states. For polymer materials, we focus more on the transition between solid and liquid, i.e., melting transition and crystallization transition, which occur in the crystalline regions in polymers. In addition, there is a unique transition that takes place in the amorphous regions of polymers -- namely the glass transition.

In crystalline regions, materials stay in the form of disordered melt at temperatures above $T_{m}$, and ordered crystalline solid below $T_{m}$. As shown in Figure \ref{fig:V vs T for Tm}, at the melting temperature $T_{m}$, the material experiences a discontinuity in the specific volume, and absorbs or releases a certain amount of heat (depending on the direction of transition), which is called latent heat. Such transitions are classified as first-order phase transitions \cite{Jaeger1998}. 

\begin{figure}[H]
	\center
	\begin{tikzpicture}[domain=0:2] 
	\begin{axis}[
	ticks=none,
	axis x line=middle,axis y line=left,
	xlabel = {$T$},
	ylabel = {$V$},
	xmin=0,xmax=9,
	ymin=0,ymax=9,ylabel style={rotate=-90},
	]
	\addplot [mark=none] coordinates {(1,1) (5,2)};
	\addplot [mark=none, dashed] coordinates {(5,2) (5,5)};
	\addplot [mark=none] coordinates {(5,5) (8,8)};
	\addplot [mark=none, red] coordinates {(5,0) (5,2)};
	\addplot [mark=none] coordinates {(5.5,0)} node[above]{$T_{m}$};
	\addplot [mark=none] coordinates {(3,1.5)} node[above]{crystal};
	\addplot [mark=none] coordinates {(6,6.5)} node[above]{liquid};
	\end{axis}
	\end{tikzpicture}
	\caption{Specific volume $V$ vs temperature $T$ of a polymer under melting or crystallization.}
	\label{fig:V vs T for Tm}
\end{figure}

In amorphous regions, materials stay in the form of a disordered liquid (viscous or rubbery) at temperatures above $T_{g}$, and transform into a disordered solid below $T_{g}$ \cite{InternationalOrganizationforStandardization2013}. At the glass transition temperature $T_{g}$, the specific volume of the material evolves continuously, and there is no latent heat involved. The physics of glass transition has not been fully understood yet, and controversy exists in describing this process. It is claimed in some theories \cite{Gibbs1958} that glass transition is a second-order phase transition, although some people consider it as a purely kinetic process \cite{Janssen2018}. Although there is no latent heat, the heat capacity of the sample does change, as indicated by the slope change in Figure \ref{fig:V vs T for Tg}. One thing to note is that the glass transition normally occurs in a range of temperatures, rather than at a single point, and it always occur below $T_{m}$. This is because the glassy state is not a thermodynamically-stable state, and the measurement of $T_{g}$ depends on factors such as the polymer's thermal history and the heating or cooling rate.

\begin{figure}[H]
	\center
\begin{tikzpicture}[domain=0:2] 
\begin{axis}[
ticks=none,
axis x line=middle,axis y line=left,
xlabel = {$T$},
ylabel = {$V$},
xmin=0,xmax=9,
ymin=0,ymax=9,ylabel style={rotate=-90},
]
\addplot [mark=none] coordinates {(1,2) (5,4)};
\addplot [mark=none] coordinates {(5,4) (8,8)};
\addplot [mark=none, red] coordinates {(5,0) (5,4)};
\addplot [mark=none] coordinates {(5.5,0)} node[above]{$T_{g}$};
\addplot [mark=none] coordinates {(3,3)} node[above]{glass};
\addplot [mark=none] coordinates {(6,6)} node[above]{liquid};
\end{axis}
\end{tikzpicture}
	\caption{Specific volume vs temperature of a polymer under glass transition, where $T$ is the temperature and $V$ is the specific volume.}
	\label{fig:V vs T for Tg}
\end{figure}

One major difference between first-order and second-order phase transitions is their driving force. In a melting transition, the process is driven by thermodynamics, as the crystalline state is the thermodynamic ground state at low temperatures. However, the glass state is not a ground state, with the chains not being fully ordered. It has been suggested in some theoretical predictions that given long enough relaxation time, the glassy state would finally transforms into the crystalline state \cite{Gotze2009}. Instead of being thermodynamically driven, the glass transition is normally considered as a kinetic transition.

\subsection{Working mechanism of differential scanning calorimetry}

Differential scanning calorimetry (DSC) is an instrument that measures the heat flow to the sample material within a controlled temperature range. Inside a typical DSC there are two metal (Al commonly) pans, with one acting as the sample pan and another empty pan acting as a reference. Through precise heating and cooling control with a feedback mechanism, the two pans are maintained at the same temperature at any time during the scanning measurement. At temperatures where phase transitions of the sample material takes place, the heat capacity of the sample changes, which requires the computer to adjust the amount of heat flow provided, in order to always keep the two pans at the same temperature. The heat flow $\dfrac{dQ}{dt}$ is obtained as a function of temperature, which depends both on the heat capacity $C_{p}$ of the sample and the scanning rate $q$:
\begin{equation}
\label{eqn_DSC}
\dfrac{dQ}{dt} = \dfrac{dQ}{dT}\cdot\dfrac{dT}{dt} = C_{p} q
\end{equation}

By plotting the difference between the heat flows to the two pans with respect to temperature, thermal transitions that the sample material experienced during the set range of temperature, such as crystallization, melting, and glass transition, can be determined. Figure \ref{fig:DSCcurveeg} is a typical DSC curve. When the scanning rate is constant, first-order transitions appear as peaks on the DSC curve. Crystallization appears as an exothermic peak on the cooling curve, and melting appears as an endothermic peak on the heating curve. The difference observed between the crystallization temperature $T_{c}$ and the melting temperature $T_{m}$ is supercooling $\Delta T$.

\begin{figure}[H]
\center
\includegraphics[width=0.5\linewidth]{DSCcurveeg}
\caption{DSC curve for the neat sample before evaporation.}
\label{fig:DSCcurveeg}
\end{figure}

\section{Results from differential scanning calorimetry}

The products previously pressed into Al pans were characterized with a DSC machine (Q100, TA Instruments). The following running process was performed on each product: equilibrate at 353 K (to fully melt all crystal); isothermal for 5 min; ramp 10 K/min to 173 K (to crystallize the sample); isothermal for 5 min; ramp 10 K/min to 353 K; isothermal for 5 min; ramp 10 K/min to 173 K; isothermal for 5 min; ramp 10 K/min to 353 K. With two runs of the same procedure, we examined the reproducibility of the results.

\subsection{Melting temperature}

From the DSC curves of each fraction, we noticed that most of the samples show a double-peak pattern, with a lower $T_{m1}$ and a higher $T_{m2}$, as shown in Figure \ref{fig:doublepeaks}. We then determined each melting temperature of every fraction, as plotted in Figure \ref{fig:Tm}.  Each measurement was carried out more than once, and the $T_{m}$ values from separate measurements normally varied within $\pm$ 2 degrees. $\bar{N}$ is the average $N$ value of each sample, which was interpolated linearly based on the 10 samples measured with MALDI. In general, the melting temperatures behave as described by the Gibbs Thomson relation, with the higher $N$ values (longer chains) showing higher melting temperatures.

\begin{figure}[H]
	\center
	\includegraphics[width=0.4\linewidth]{doublepeaks}
	\caption{Double-peak pattern observed on the DSC heating run of a purified fraction.}
	\label{fig:doublepeaks}
\end{figure}

\begin{figure}[H]
\center
\includegraphics[width=0.6\linewidth]{Tm}
\caption{Melting temperature of purified fractions.}
\label{fig:Tm}
\end{figure}

\subsubsection{Chain-folding analysis based on $T_{m}$}

In Figure \ref{fig:Tm}, it is obviously seen that the data points potentially lie on two roughly parallel curves, which brings our assumption that they could correspond to two types of chain-folding modes in the crystal lamellae, with the higher $T_{m}$'s being extended chains (larger thickness), and the lower $T_{m}$'s being once-folded chains (smaller thickness).

In order to validate our assumption, we apply Equation \ref{eqn_GT} to see if we are able to get a good fit with the two series of data. Parameters for PEO present in this equation, including $T_{m}^{\infty}$ \cite{Buckley1975}, $V$ \cite{Wong2015}, $\Delta H$ \cite{Pielichowski2002} are found in the literature. Interfacial tension $S$ is dependent on the mode of chain-folding, as both chain ends and chain folds contribute to the amorphous phase, and they lead to different interfacial tensions with respect to the crystalline phase.

The interfacial tension of chain folds, $S_{folds}$, can be obtained from parameters of PEO chains with large molecular weights. This is because in the crystal lamellae of long chains, the number of chain folds are much greater than that of chain ends, and thus $S$ is dominated by chain folds. For long PEO chains, crystal lamellar thickness $L$ is normally on the order of 10 nm \cite{Okerberg2007}, and the melting temperature of high molecular weight PEO is around $65^\circ$C \cite{Herzberger2015}. With the other parameters previously found, we are able to calculate $S_{folds} = 98.4 mJ/m^2$ from Equation \ref{eqn_GT}.

However, to quantitatively look at the thermodynamics of extended chains and once-folded chains in our assumption, and to fit the Gibbs Thomson relation of these two modes to our data, we need to know the actual interfacial tensions in these two modes. 

For extended chains, the interfacial tension $S_{ext}$ merely comes from chain ends, while for once-folded chains, apart from the chain ends, there are also chain folds that contribute to the interfacial tension $S_{1-fold}$. $S_{ext}$ and $S_{1-fold}$ can be obtained by adjusting their values based on $S_{folds}$ (previously calculated for long chains). The reason we are able to do this is that even though they arise from different parts in the polymer, the interfacial tension between crystalline and amorphous regions should not vary significantly (at least on the same scale) for a certain polymer. The following two figures and illustrations describe how we achieved our fitting and established our model on the conformation of chains.

For extended chains, we fit the higher melting points with the Gibbs Thomson equation, using the value of $S_{folds}$ initially, and then adjust its value until we get a good enough fit (Figure \ref{fig:fitTm1}). This value is then taken as $S_{ext} = 275.1 mJ/m^2$.

\begin{figure}[H]
\center
\includegraphics[width=0.6\linewidth]{fitTm1}
\caption{$T_{m1}$ data fitting to Gibbs Thomson equation.}
\label{fig:fitTm1}
\end{figure}

For the lower melting points, which correspond to once-folded chains, the chains emanating from the lamella enter the amorphous phase to make a fold, and then re-enter the lamella, where 3.5 monomers are needed for a single chain to complete this turn, as introduced in Section \ref{crystalstructure}. This conformation then has two chain-end monomers on one side of the lamella, and 3.5 monomers on the fold on the other side, which enables us to calculate the interfacial tension $S_{1-fold}$ as:

\begin{equation}
\label{eqn_S1fold}
S_{1-fold} = \dfrac{2S_{ext} + 3.5 S_{folds}}{2 + 3.5} = 162.7 mJ/m^2
\end{equation}

With every parameter in the Gibbs Thomson equation obtained, we then generated the curve of melting temperature of once-folded chains as a function of $N$ values. The good agreement between the curve and the real data points (Figure \ref{fig:fitTm2}) suggests that this is a possible and reasonable model of chain conformation.

\begin{figure}[H]
\center
\includegraphics[width=0.6\linewidth]{fitTm2}
\caption{$T_{m1}$ and $T_{m2}$ data fitting to Gibbs Thomson equation.}
\label{fig:fitTm2}
\end{figure}

In the plots of melting temperatures, the x-axis, $\bar{N}$, is the number average value of all the composing $N$'s in each fraction, characterized directly with MALDI or interpolated based on the MALDI data. However, only with single integer $N$ values could we be able to talk about the melting temperatures given by the Gibbs Thomson curves. For a mixture of different $N$'s, its melting temperature potentially lies anywhere within the range of $T_{m}$'s of its composing $N$'s. A more careful way to present our chain-folding models together with the $T_{m}$ data would be as Figure \ref{fig:fitTm} shows. Dashed boxes are generated for each $T_{m}$ curve, with the top (bottom) of the box representing the melting point of the highest (lowest) $N$ value present in any potential purified fraction lying on the curve in this particular box. In generating the bars, $N$ components with a percentage less than 5 \% are neglected. Notice that each curve passes through all of the corresponding boxes, with only a few data points falling outside.

\begin{figure}[H]
\center
\includegraphics[width=0.6\linewidth]{fitTm}
\caption{Gibbs Thomson relation fitting with potential range bars on $T_{m}$ data points.}
\label{fig:fitTm}
\end{figure}

\subsubsection{Comparison to N-alkanes}

As mentioned previously, PEO is a polymer with linear structure, as well as N-alkanes and polyethylene. Therefore, some comparisons to N-alkanes are necessary to better understand the phenomena and properties we have observed.

Before the lamellar structure of PEO was discovered, N-alkanes had been revealed to form "single crystal platelets", with the chain ends at the surface of these lamellae \cite{Richardson1965}. The thickness of each lamella was around 100 \AA, while it could vary from 60-80 \AA \ to up to 150 \AA, depending on certain conditions \cite{Keller1957a}. In order for long chains whose extended length was larger than this thickness to fit in such a layer, polymers were found to fold back and forth in the layer. Depending on the length of chains, they needed to fold for different times in the lamellae. For N-alkanes, the lower molecular weight limit for fold-chain conformation was 2100 $g/mol$, or 150 carbon atoms in the chain \cite{Ungar1986}. For crystallization from solutions, this limit was slightly lower than that from the melt, but still similar \cite{Alamo1993}.

It was further discovered that for long N-alkanes, the fold length (lamella thickness) was a function of crystallization temperature $T_{c}$. When the supercooling was small, lamellae grew with larger thickness \cite{Ungar1986}. 

The number of folds for each polymer chain was at first believed to be quantized, i.e., the chains could only take integer number of folds in the lamella, or they existed as extended chains \cite{Ungar1986}. Kovacs \textit{et al} obtained same results for PEO oligomers as well, where only integer folds were allowed \cite{Kovacs1977}. However, it was later discovered that non-integer folds (NIF) were also possible. Real-time small-angle X-ray scattering (SAXS) experiments \cite{Zeng1998} revealed that at early stages of crystal formation, long alkane chains formed NIF crystals. The amorphous layers in between of lamellae of NIF crystals had a thickness of 6 to 8 nm, which were much looser than those of extended chain crystals. During crystallization, NIF lamellae further thickened or thinned until the thickness reached integral fractional (IF) values of the extended chain length, through refolding of chains. The amorphous layers then became denser and the folds turned sharper, corresponding to a more stable state of the crystal \cite{Ungar1986}.

Richardson \cite{Richardson1965} studied single crystal polyethylene with an adiabatic calorimeter, in order to investigate the folding and chain re-entry in the lamellae. Chains exiting the crystalline layer might return to themselves immediately, as suggested in the adjacent re-entry model in Section \ref{sec: folded chain model}, or they might float in the amorphous region, and return to the lamellae from a further site, resulting in a loose fold. From the result of calorimetry and small-angle X-ray experiments, the number of carbon atoms involved in each sharp fold in the adjacent re-entry model was found to be six, which is three monomers for polyethylene. For low molecular weight PEO, the number of monomers on a fold is 3.5 on average, which is comparable with polyethylene.

\subsubsection{Fractionation of chains during crystallization}

In the DSC measurements, fractionation of chains with different $N$'s is sometimes observed. During some of the repeated DSC measurements, several samples showed double melting peaks (an example shown in Figure \ref{fig:3K difference}), with both melting temperatures near the same $T_{m}$ curve. The two peaks were separated by around 3 K, which is likely the difference between the melting temperatures of two neighbouring $N$'s according to our calculation, rather than the difference between the two chain conformations (extended and folded). It is worth noting that this fractionation is more commonly observed at the higher $T_{m}$ than at the lower $T_{m}$, because in the extended conformation, the crystal lamellae composing of two neighbouring $N$'s have a larger difference in thickness than in the folded conformation.

\begin{figure}[H]
	\center
	\includegraphics[width=0.4\linewidth]{doublepeaks3K}
	\caption{Part of DSC curve (melting) from regular run on $\bar{N} = 11$.}
	\label{fig:3K difference}
\end{figure}

\subsubsection{Tuning chain-folding mode}

Most of the fractions showed two melting points in DSC measurements, while some of the fractions only showed one, lying either on the $T_{m1}$ curve or the $T_{m2}$ curve. For the fractions where the higher $T_{m}$ was observed, on the DSC cooling ramp usually showed two crystallization peaks, suggesting that the polymers still formed both extended and folded chain structures, but before increasing to the melting temperature, once-folded chains relaxed themselves and recrystallized into extended form. However, when the lower $T_{m}$ was present, only one crystallization peak was observed on the This is an indication that the cooling rate during crystallization might not have been slow enough for the chains to crystallize in the extended form.

\begin{figure}[H]
	\center
	\includegraphics[width=0.4\linewidth]{treatment}
	\caption{Thermal treatment on products with a lower $T_{m}$ present.}
	\label{fig:treatment}
\end{figure}

The following treatment (as shown in Figure \ref{fig:treatment}) was then applied to further verify our observation. With some of the fractions that showed the lower $T_{m}$ (either with or without the higher $T_{m}$) in normal DSC measurements, we kept the sample at a temperature between $T_{m1}$ and $T_{m2}$ for a time long enough to melt all the once-folded chains and leave all the extended chains. Then we quickly cooled the sample to a much lower temperature, and measured its melting again. During the second DSC measurement only the higher $T_{m}$ appeared, which was a direct validation that we had successfully forced the once-folded chains to recrystallize into extended chains by applying the treatment. Figure \ref{fig:DSC before and after} is an example measurement we did on a purified fraction with $N = 12.3$. This treatment and the result we obtained is again a proof of the existence of folded chain configurations in the crystal lamellae. However, as introduced previously, PEO chains have been believed to fold only when $N \geq 45$. Therefore, this is a surprising result to us, and further investigations are needed to reveal the mechanism behind it.

\begin{figure}[H]
	\centering
\begin{subfigure}[b]{0.5\linewidth}
	\includegraphics[width=\linewidth]{DSCbefore}
	\caption{Before thermal treatment there is a major melting transition peak followed by a small peak.}
\end{subfigure}
\begin{subfigure}[b]{0.5\linewidth}
	\includegraphics[width=\linewidth]{DSCafter}
	\caption{After thermal treatment only one melting peak near $T_{m2}$ is observed.}
\end{subfigure}
\caption{DSC measurements on a purified fraction with $N = 12.3$ before and after thermal treatment.}
\label{fig:DSC before and after}
\end{figure}

\subsubsection{Comparison between purified fractions and neat sample}

When we compared the DSC curves of the neat sample and the purified fractions, some of the fractions showed unexpected melting behaviours (Figure \ref{fig:neatvsfractions}). From the curves of some products obtained at early stages of evaporation, the neat mixed sample should start melting at temperatures much lower than the observed melting temperature of the neat sample. However, this was not observed. Instead, we believe that when the short chains are mixed with longer chains, they become influenced and crystallizes differently than in a more monodisperse sample. One of the possible explanations to this is that in the presence of longer chains, the shorter chains tend to act as the amorphous phase, even at temperatures lower than their own melting points. It is also possible that the short chains only represent a very small portion (fractions with $N \leq 10$ are 14\% of the neat sample) of the neat sample. Therefore the heat flow signal from them could easily be overwhelmed by that of major components.

\begin{figure}[H]
\center
\includegraphics[width=0.6\linewidth]{neatvsfractions}
\caption{DSC curves of the neat sample (red) and some of the purified fractions (blue).}
\label{fig:neatvsfractions}
\end{figure}

\subsubsection{End-group effects}

In \ref{Tm and Yeates}, we reviewed Yeates' study on the melting of monodisperse and polydisperse PEO oligomers. Now we would like to compare our results for purified fractions with theirs. Surprisingly, it turns out that the melting temperatures we obtained agreed more with the polydisperse samples in their measurements. This could be an indication that the melting points difference they observed between the monodisperse and polydisperse samples is very unlikely to be due to polydispersity. Instead, it could be related to specific properties such as end-group chemistry.

End-group effects on PEO crystallization has been studied by many researchers and it has been found that the type of end-group directly influences properties including melting temperature and crystallinity, as shown in Table \ref{tab:end-group effects} \cite{Marshall1981}. Monodisperse PEO with hydroxy end-groups has been reported to display different crystallinities and $T_{m}$'s than that with methoxy end-groups. The difference in crystallinity is believed to be due to different heats of interaction related to the end-groups at the lamellar surfaces. The difference in melting temperature is attributed to different environments at a crystalline lamellar surface and in melt, because lamellar surfaces are much more ordered compared to the melt, which magnifies the effect of end-groups on the surfaces, while in a melt, the effect of end-groups could be hidden in the melt background. Polydisperse PEO samples with different end-groups, however, display different crystallinities but similar $T_{m}$'s. It is argued that rejection of methoxy end-groups from the lamellar surfaces results in higher entropy of the crystal, leading to a lower enthalpy of melting, and a lower crystallinity. In terms of the melting temperature, it is claimed that the disordered lamellar surface and the melt have similar environments, so the effect of end-groups on $T_{m}$'s would appear less significant.
\begin{table}[H]
\centering
	\begin{tabular}{ |c|c|c|c| } 
		\hline
		 & monodisperse PEO & polydisperse PEO \\
		\hline
		\hline
		crystallinity & different & different \\ 
		\hline
		$T_{m}$ & different & similar \\ 
		\hline
	\end{tabular}
	\caption{\label{tab:end-group effects}Comparison of PEO with hydroxy and methoxy end-groups (reproduced from \cite{Marshall1981}).}
\end{table}
In our experiments, the PEO samples only contain hydroxy end-groups, while in Yeates' study, the synthesis of monodisperse samples involved end-groups containing sulfur. Based on the evidence and analysis mentioned above, the  disagreement between our results and theirs could be that sulfur results in different interaction energy with the crystalline layer, and potentially led to different melting temperatures. Sulfur has been found to decrease the interfacial tension of liquid iron with $\text{Al}_{\text{2}}\text{O}_{\text{3}}$ \cite{Halden1955}, and also decrease the interfacial energy between Fe-C melt and graphite \cite{Jung2005}. However, no direct measurement results have been found in terms of the effect of sulfur-containing end-groups on the interfacial energy and melting temperature of a polymer.

\subsection{Degree of crystallinity}

PEO has been known as a polymer with high crystallinity due to its linear structure. However, based on the fact that polymers almost never crystallize completely, it is of interest to study the degree of crystallinity $X_{c}$ of our samples. We determined $X_{c}$ of the products from the heat of fusion $\Delta H_{f} (T_{m})$ on melting in DSC measurements. The heat of fusion can be calculated from the area under the melting peak, and the degree of crystallinity was defined as \cite{Pielichowski2002}:

\begin{equation}
\label{eqn_Xc}
X_{c} = \dfrac{\Delta H_{f} (T_{m})}{\Delta H_{f}^{0} (T_{m}^{0})}
\end{equation}

\noindent
where $X_{c}$ is the degree of crystallinity by weight fraction, $\Delta H_{f} (T_{m})$ is the enthalpy of the melting transition, and $\Delta H_{f}^{0} (T_{m}^{0})$ is the enthalpy of melting of PEO with 100 \% crystallinity \cite{Pielichowski2002}. By integrating the melting peaks on the DSC curves, we obtain $X_{c}$ of the purified products, as shown in Figure \ref{fig:Xc}.

\begin{figure}[H]
\center
\includegraphics[width=0.5\linewidth]{Xc}
\caption{Degree of crystallinity of purified products.}
\label{fig:Xc}
\end{figure}

It is easily noticed from the figure that our data is not precise enough since the error bars for some of the fractions are quite large. This results from the deviation (0.1 mg) of the scale used to weigh the samples. The enthalpy of melting $\Delta H_{f} (T_{m})$ calculated from the DSC curve is directly related to the weight of sample, and for samples with a small weight, the deviation is comparable to its weight. Therefore, for more precise measurement of $X_{c}$, larger amounts of sample are required, which brings forward the demand for technical improvements including scaling-up of our evaporative purification system.

We would also like to compare our results of PEO crystallinity with other studies, as shown in Figure \ref{fig:Xc comparison}. Although there are a limited number of measurement results on PEO crystallinity with such low $\bar{N}$ values, most of our results lie in the range of several other studies in the literature.

\begin{figure}[H]
	\center
	\includegraphics[width=0.6\linewidth]{Xccomparison}
	\caption[Comparison of degree of crystallinity measured in different studies]{Comparison of degree of crystallinity measured in different studies. Majumdar: data from "Physical characterization of polyethylene glycols by thermal analytical technique and the effect of humidity and molecular weight" by R Majumdar \textit{et al}, \textit{Pharmazie}, 65:343-347, 2010 \cite{Majumdar2010}. Marshall: data from "Crystallinity of Ethylene Oxide Oligomers" by A Marshall \textit{et al}, \textit{Eur. Polym. Journal}, 17(893), 1981. \cite{Marshall1981}.}
	\label{fig:Xc comparison}
\end{figure}

An interesting fact about the $X_{c}$ data is that when we bring Figure \ref{fig:Xc} and Figure \ref{fig:Tm} into comparison (even though $X_{c}$ and $T_{m}$ are not directly related), it was noticed that they have a similar trend, especially with the bump pattern located at $\bar{N}$ around 11. However, the reason behind this observation is still unclear to us and needs further investigation.