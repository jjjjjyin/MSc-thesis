\chapter{Conclusion \& discussion}

To conclude, the three chapters in the main body are the three major investigations conducted with oligomeric PEO: evaporative purification, chain-conformation, and crystal growth. We applied the evaporative purification technique to obtain highly monodisperse PEO oligomers. The purified products have been examined with mass spectroscopy, in order to characterize the distribution of N’s. With DSC, we studied their crystallization and melting behaviors.

\section{Evaporative purification of PEO}

Through evaporative purification, the distribution of purified samples gets significantly narrowed down compared to that of the neat sample. Quantitatively the polydispersity is approximately six time better, according to the MALDI results. The evolution curves of different $N$ values intuitively show that specific distribution of $N$’s could be assigned correspondingly to the evaporation time or temperature, which provides the possibility of commercializing this purification technique for low molecular weight polymers. In order to further reduce PDI, some practical improvements include: applying larger scale of neat sample; reducing collection intervals; carrying out multiple cycles of evaporation. As a matter of fact, PEO is not a perfect material in terms of this evaporation technique, as the vapor pressures of different $N$ values are not well separated. Comparing to those of polystyrene \cite{Zhu2017a}, the gaps between each two vapor pressure curves of PEO are smaller, which results in more difficult separation through evaporation.

\section{Chain conformation}

The full extended length of the largest $N$ we obtained ($N$ = 16) is less than 5 nm, while polymer crystal lamella is commonly on the order of 10 nm in thickness \cite{Savage2015}. Although the $N$ values we have obtained are still considered small, and normally not expected to be able to fold, the best model to describe our $T_{m}$ data from DSC measurements is that the higher $T_{m}$’s adopt the extended chain mode, and the lower $T_{m}$’s adopt the once-folded chain mode. Furthermore, the fact that we are able to eliminate the lower $T_{m}$ and generate only the higher $T_{m}$ through thermal treatment proves that we could tune the chain-folding mode from extended chains to once-folded chains, which is a direct validation of our model. For common commercial PEO oligomer samples, PDI is high, and the distribution over different chain lengths is broad, which potentially makes it difficult for all of them to make a once-fold and form an ordered conformation in the lamella. However, when chain lengths are all similar, it is possible that the entropy of a once-folded lamella surface is lower, which makes this conformation easier to exist. The measurement of crystallinity was limited by the instrumentations, and thus a larger scale of evaporation could also benefit crystallinity measurement.

\section{Crystal growth}

Processes, kinetics, and crystal structures in polymer crystal growth are reviewed in Chapter \ref{chap_growth}. Macrostructures named spherulites are formed in polymer crystal growth, and the measurements of PEO crystal growth rates are conducted based on the measurements of spherulite size under an optical microscope. The fact that nucleation rate is much slower than crystal growth rate enables us to conveniently carry out the growth rate measurement independently. Due to the weak contrast at temperatures near $T_{m}$'s, we have only obtained data for PEO crystallization under large supercoolings, which agree well with the data from low temperature regions in the literature. In future work, regions nearer melting temperatures should be examined. In addition, X-ray experiments would also be an ideal way to look further into the crystal lamellae.