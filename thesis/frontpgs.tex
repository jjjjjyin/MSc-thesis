% T I T L E   P A G E
% -------------------

\pagestyle{empty}
\pagenumbering{roman}

\begin{titlepage}
        \begin{center}
        \vspace*{1.0cm}

        \Huge
        {\bf Crystallization Studies of Highly Monodisperse Oligomeric Poly(Ethylene Oxide) }

        \vspace*{1.0cm}

        \normalsize
        by \\

        \vspace*{1.0cm}

        \Large
        Junjie Yin \\

        \vspace*{3.0cm}

        \normalsize
        A thesis \\
        presented to the University of Waterloo \\ 
        in fulfillment of the \\
        thesis requirement for the degree of \\
        Master of Science \\
        in \\
        Physics (Nanotechnology)\\

        \vspace*{2.0cm}

        Waterloo, Ontario, Canada, 2018 \\

        \vspace*{1.0cm}

        \copyright\ Junjie Yin \\
        \end{center}
\end{titlepage}

% The rest of the front pages should contain no headers and be numbered using Roman numerals starting with `ii'
\pagestyle{plain}
\setcounter{page}{2}

\cleardoublepage % Ends the current page and causes all figures and tables that have so far appeared in the input to be printed.
% In a two-sided printing style, it also makes the next page a right-hand (odd-numbered) page, producing a blank page if necessary.
 
\doublespacing

% D E C L A R A T I O N   P A G E
% -------------------------------
  % The following is the sample Delaration Page as provided by the GSO
  % December 13th, 2006.  It is designed for an electronic thesis.
  \noindent
I hereby declare that I am the sole author of this thesis. This is a true copy of the thesis, including any required final revisions, as accepted by my examiners.

  \bigskip
  
  \noindent
I understand that my thesis may be made electronically available to the public.

\cleardoublepage
%\newpage

% A B S T R A C T
% ---------------

\begin{center}\textbf{Abstract}\end{center}

Poly(ethylene oxide) is one of the most intensively studied polymers in terms of crystallization, because of its linear structure. In this thesis the chapters are organized in a self-contained fashion, with a general introduction and a brief conclusion. We introduce the purification and characterization of highly monodisperse PEO oligomers, and the analysis of their melting and crystallization behaviours. Through evaporative purification, we have been able to purify low molecular weight PEO, and achieve a polydispersity index six times better than the neat commercial sample, as measured by mass spectroscopy. Melting temperatures are obtained using differential scanning calorimetry. Based on the Gibbs Thomson relation, we claim that during crystallization, some purified PEO samples can form crystal lamellae not only with extended chains, but also with once-folded chains, which is normally not expected for polymers with such short chain lengths. The fact that we are able to control the melting temperature through annealing treatment on the crystal validates our chain-folding model of folded and extended chains.

\cleardoublepage
%\newpage

% A C K N O W L E D G E M E N T S
% -------------------------------
\begin{center}\textbf{Acknowledgements}\end{center}

Firstly, I would like to express my sincere gratitude to my supervisor James A. Forrest for his guidance and support in the past two years. He has been an inspirational mentor to me since I first stepped into the world of scientific research. Every discussion with Jamie opens my mind and fuels my motivations to pursue science. Thank you Jamie for being an excellent supervisor.

I would also like to thank my advisory committee members Jean Duhamel and David Hawthorn, who have provided support and helpful suggestions on my research project. I am also grateful to Jean Duhamel and Bae-Yeun Ha, for being on my defence committee, and for their time and effort in reviewing my thesis.

My fellow students and friends in the lab have made my master's study and research a great experience. Adam Raegen, our knowledgeable postdoc in the lab, has been a teacher to me. I have learned useful experimental skills and research methods from him. I would like to thank all former and present members in our lab who have worked with me. Tiana Trumpour, Shipei Zhu, Valentin Ruffine, Neha Dhalwani, Denzil Barkley, thank you for the discussions and talks we have had and the help you provided me. It has been an enjoyable thing to work with these wonderful people.

I am also thankful to Stefan Idziak, who has a long-time collaboration with our lab, and has helped me with experiments in my project. I would like to thank our administrative staff Judy McDonnell, Bonnie Findlay in Physics, and Lisa Pokrajac in WIN for their help. I would also like to thank all the friends I have met in Waterloo. Thank you all for making my life here joyful.

Lastly, I would like to thank my dearest family. Thank my parents Baocai Yin and Ziping Wang for their unconditional love and encouragement throughout my life. Thank my siblings for their kindness. My thanks also goes to my boyfriend Junan Lin, for his love and care, and all the helpful discussions we have had. Thank you all for everything. You mean so much in my life.

\cleardoublepage
%\newpage


% T A B L E   O F   C O N T E N T S
% ---------------------------------
% To define the depth that the table of contents goes, set the following in the preamble:
%\setcounter{tocdepth}{1} % Show sections
%\setcounter{tocdepth}{2} % + subsections
%\setcounter{tocdepth}{3} % + subsubsections
%\setcounter{tocdepth}{4} % + paragraphs
%\setcounter{tocdepth}{5} % + subparagraphs

\renewcommand\contentsname{Table of Contents}
\tableofcontents
\cleardoublepage
\phantomsection
%\newpage

% L I S T   O F   T A B L E S
% ---------------------------
%\addcontentsline{toc}{chapter}{List of Tables}
%\listoftables
%\cleardoublepage
%\phantomsection		% allows hyperref to link to the correct page
%\newpage

% L I S T   O F   F I G U R E S
% -----------------------------
\addcontentsline{toc}{chapter}{List of Figures}
\listoffigures
\cleardoublepage
\phantomsection		% allows hyperref to link to the correct page
%\newpage

% L I S T   O F   S Y M B O L S
% -----------------------------
% To include a Nomenclature section
% \addcontentsline{toc}{chapter}{\textbf{Nomenclature}}
% \renewcommand{\nomname}{Nomenclature}
% \printglossary
% \cleardoublepage
% \phantomsection % allows hyperref to link to the correct page
% \newpage

% Change page numbering back to Arabic numerals
\pagenumbering{arabic}

