% T I T L E   P A G E
% -------------------

\pagestyle{empty}
\pagenumbering{roman}

\begin{titlepage}
        \begin{center}
        \vspace*{1.0cm}

        \Huge
        {\bf Production and Analysis of Highly Monodisperse Oligomeric Poly(Ethylene Oxide) }

        \vspace*{1.0cm}

        \normalsize
        by \\

        \vspace*{1.0cm}

        \Large
        Junjie Yin \\

        \vspace*{3.0cm}

        \normalsize
        A thesis \\
        presented to the University of Waterloo \\ 
        in fulfillment of the \\
        thesis requirement for the degree of \\
        Master of Science \\
        in \\
        Physics (Nanotechnology)\\

        \vspace*{2.0cm}

        Waterloo, Ontario, Canada, 2018 \\

        \vspace*{1.0cm}

        \copyright\ Junjie Yin 2018 \\
        \end{center}
\end{titlepage}

% The rest of the front pages should contain no headers and be numbered using Roman numerals starting with `ii'
\pagestyle{plain}
\setcounter{page}{2}

\cleardoublepage % Ends the current page and causes all figures and tables that have so far appeared in the input to be printed.
% In a two-sided printing style, it also makes the next page a right-hand (odd-numbered) page, producing a blank page if necessary.
 
\doublespacing

% D E C L A R A T I O N   P A G E
% -------------------------------
  % The following is the sample Delaration Page as provided by the GSO
  % December 13th, 2006.  It is designed for an electronic thesis.
  \noindent
I hereby declare that I am the sole author of this thesis. This is a true copy of the thesis, including any required final revisions, as accepted by my examiners.

  \bigskip
  
  \noindent
I understand that my thesis may be made electronically available to the public.

\cleardoublepage
%\newpage

% A B S T R A C T
% ---------------

\begin{center}\textbf{Abstract}\end{center}

Apart from Poly(Ethylene) and n-alkanes, Poly(Ethylene Oxide) is one of the most intensively studied polymers in terms of crystallization, because of its linear structure. In this thesis, the chapters are organized in a self-contained fashion, investigating three different aspects of PEO oligomers, with a general introduction and a brief conclusion. We introduce the production of highly monodisperse PEO oligomers, and the analysis of their crystallization and melting behaviors. Through evaporative purification, we have been able to purify low molecular weight PEO oligomers, and achieve a polydispersity index six times better than the neat commercial sample, measured by mass spectroscopy. Melting temperatures are obtained using differential scanning calorimetry. Based on Gibbs Thomson relation, we claim that during crystallization, the purified PEO samples can form crystal lamellae not only with extended chains, but also with once-folded chains, which is normally not expected for polymers with such short chain lengths. 3.5 monomers are required to complete each fold, as suggested by the fitting of Gibbs Thomson curve. The fact that we are able to control the melting temperature validates our chain-folding model of folded and extended chains.

\cleardoublepage
%\newpage

% A C K N O W L E D G E M E N T S
% -------------------------------

\begin{center}\textbf{Acknowledgements}\end{center}


\cleardoublepage
%\newpage

% T A B L E   O F   C O N T E N T S
% ---------------------------------
% To define the depth that the table of contents goes, set the following in the preamble:
%\setcounter{tocdepth}{1} % Show sections
%\setcounter{tocdepth}{2} % + subsections
%\setcounter{tocdepth}{3} % + subsubsections
%\setcounter{tocdepth}{4} % + paragraphs
%\setcounter{tocdepth}{5} % + subparagraphs

\renewcommand\contentsname{Table of Contents}
\tableofcontents
\cleardoublepage
\phantomsection
%\newpage

% L I S T   O F   T A B L E S
% ---------------------------
%\addcontentsline{toc}{chapter}{List of Tables}
%\listoftables
%\cleardoublepage
%\phantomsection		% allows hyperref to link to the correct page
%\newpage

% L I S T   O F   F I G U R E S
% -----------------------------
\addcontentsline{toc}{chapter}{List of Figures}
\listoffigures
\cleardoublepage
\phantomsection		% allows hyperref to link to the correct page
%\newpage

% L I S T   O F   S Y M B O L S
% -----------------------------
% To include a Nomenclature section
% \addcontentsline{toc}{chapter}{\textbf{Nomenclature}}
% \renewcommand{\nomname}{Nomenclature}
% \printglossary
% \cleardoublepage
% \phantomsection % allows hyperref to link to the correct page
% \newpage

% Change page numbering back to Arabic numerals
\pagenumbering{arabic}

